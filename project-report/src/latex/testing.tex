\documentclass[./project-report/src/latex/project-report.tex]{subfiles}

\begin{document}

\maketitle

\clearpage
\section{Testing}

Testing on the project source code consists of manual testing, where the inputs to frames were tested to ensure correct behaviour and error handling, and automated testing 
through the use of unit tests that run upon every commit to GitHub.

\subsection{Summary of tests}

\begin{tabular}{|p{0.15\linewidth}|p{0.76\linewidth}|p{0.09\linewidth}|}
	\hline
	\textbf{Test Type} & \textbf{Description} & \textbf{Result} \\
	\hline
	Manual & Hyper Parameter Frame - Use GPU Validation & Pass \\
	\hline
	Manual & Hyper Parameter Frame - Non-Numeric Hidden Layers Shape Validation & Pass \\
	\hline
	Manual & Hyper Parameter Frame - Negative Hidden Layers Shape Validation & Pass \\
	\hline
	Manual & Hyper Parameter Frame - Invalid Delimiter Hidden Layers Shape Validation & Pass \\
	\hline
	Manual & Load Model Frame - Use GPU Validation & Pass \\
	\hline
	Manual & Test Frames - Taken Trained Model Name Validation & Pass \\
	\hline
	Manual & Test Frames - Empty Trained Model Name Validation & Pass \\
	\hline
	Manual & Test Frames - Invalid Delimiter Trained Model Name Validation & Pass \\
	\hline
	Unit & test\_database.py - test\_database\_structure & Pass \\
	\hline
	Unit & test\_database.py - test\_not\_null\_constraint & Pass \\
	\hline
	Unit & test\_database.py - test\_unique\_constraint & Pass \\
	\hline
	Unit & test\_database.py - test\_save\_load\_consistency & Pass \\
	\hline
	Unit & test\_model.py - test\_train\_dataset\_size & Pass \\
	\hline
	Unit & test\_model.py - test\_network\_shape & Pass \\
	\hline
	Unit & test\_model.py - test\_learning\_rates & Pass \\
	\hline
	Unit & test\_model.py - test\_relu\_model\_transfer\_types & Pass \\
	\hline
	Unit & test\_model.py - test\_sigmoid\_model\_transfer\_types & Pass \\
	\hline
	Unit & test\_model.py - test\_weight\_matrice\_shapes & Pass \\
	\hline
	Unit & test\_model.py - test\_biase\_matrice\_shapes & Pass \\
	\hline
	Unit & test\_model.py - test\_layer\_output\_shapes & Pass \\
	\hline
	Unit & test\_model.py - test\_save\_model & Pass \\
	\hline
	Unit & test\_tools.py - test\_relu & Pass \\
	\hline
	Unit & test\_tools.py - test\_sigmoid & Pass \\
	\hline
\end{tabular}

\subsection{Manual Testing - Input Validation Testing}

The following tests check the input validation of each frames' inputs.

\subsubsection{Hyper Parameter Frame}
\label{sec:hyper-parameter-frame-input-validation}
\begin{itemize}
	\item Use GPU Validation: \newline\newline
		\begin{tabular}{|p{0.25\linewidth}|p{0.75\linewidth}|}
			\hline
			Description & Select Use GPU checkbox without a GPU present. \\
			\hline
			Expected Result & The exception should be handled and a useful error message should be displayed. \\
			\hline
			Actual Result & Expected Result \\
			\hline
			Test Status & Pass \\
			\hline
		\end{tabular}

		\vspace{5mm}

		Evidence:
		\begin{figure}[h!]
		\centering
		\frame{\includegraphics[width=1\textwidth]{./project-report/src/images/create-model-use-gpu-validation.png}}
		\caption{Use GPU Validation evidence}
		\end{figure}

		Link to video evidence: \url{https://github.com/mcttn22/school-project/blob/main/project-report/testing-videos.md/#use-gpu-validation}

	\pagebreak

	\item Non-Numeric Hidden Layers Shape Validation: \newline\newline
		\begin{tabular}{|p{0.25\linewidth}|p{0.75\linewidth}|}
			\hline
			Description & Enter a non-numeric hidden layers shape. \\
			\hline
			Data Value & "test" \\
			\hline
			Data Type & Erroneous \\
			\hline
			Expected Result & The exception should be handled and a useful error message should be displayed. \\
			\hline
			Actual Result & Expected Result \\
			\hline
			Test Status & Pass \\
			\hline
		\end{tabular}

		\vspace{5mm}

		Evidence:
		\begin{figure}[h!]
		\centering
		\frame{\includegraphics[width=1\textwidth]{./project-report/src/images/non-numeric-hidden-layers-shape-input-validation.png}}
		\caption{Non-Numeric Hidden Layers Shape Validation evidence}
		\end{figure}

		\begin{sloppypar}
		Link to video evidence: \url{https://github.com/mcttn22/school-project/blob/main/project-report/testing-videos.md/#non-numeric-hidden-layers-shape-validation}
		\end{sloppypar}

		\pagebreak

	\item Negative Hidden Layers Shape Validation: \newline\newline
		\begin{tabular}{|p{0.25\linewidth}|p{0.75\linewidth}|}
			\hline
			Description & Enter a negative hidden layers shape. \\
			\hline
			Data Value & "-100" \\
			\hline
			Data Type & Erroneous \\
			\hline
			Expected Result & The exception should be handled and a useful error message should be displayed. \\
			\hline
			Actual Result & Expected Result \\
			\hline
			Test Status & Pass \\
			\hline
		\end{tabular}

		\vspace{5mm}

		Evidence:
		\begin{figure}[h!]
		\centering
		\frame{\includegraphics[width=1\textwidth]{./project-report/src/images/negative-hidden-layers-shape-input-validation.png}}
		\caption{Negative Hidden Layers Shape Validation evidence}
		\end{figure}

		\begin{sloppypar}
		Link to video evidence: \url{https://github.com/mcttn22/school-project/blob/main/project-report/testing-videos.md/#negative-hidden-layers-shape-validation}
		\end{sloppypar}

	\pagebreak

	\item Invalid Delimiter Hidden Layers Shape Validation: \newline\newline
		\begin{tabular}{|p{0.25\linewidth}|p{0.75\linewidth}|}
			\hline
			Description & Enter a hidden layers shape with invalid delimiters. \\
			\hline
			Data Value & "100,,100" \\
			\hline
			Data Type & Erroneous \\
			\hline
			Expected Result & The exception should be handled and a useful error message should be displayed. \\
			\hline
			Actual Result & Expected Result \\
			\hline
			Test Status & Pass \\
			\hline
		\end{tabular}

		\vspace{5mm}

		Evidence:
		\begin{figure}[h!]
		\centering
		\frame{\includegraphics[width=1\textwidth]{./project-report/src/images/invalid-delimiter-hidden-layers-shape-input-validation.png}}
		\caption{Invalid Delimiter Hidden Layers Shape Validation evidence}
		\end{figure}

		\begin{sloppypar}
		Link to video evidence: \url{https://github.com/mcttn22/school-project/blob/main/project-report/testing-videos.md/#invalid-delimiter-hidden-layers-shape-validation}
		\end{sloppypar}
\end{itemize}

\pagebreak

\subsubsection{Load Model Frame}
\label{sec:load-model-frame-input-validation}
\begin{itemize}
	\item Use GPU Validation: \newline\newline
		\begin{tabular}{|p{0.25\linewidth}|p{0.75\linewidth}|}
			\hline
			Description & Select Use GPU checkbox without a GPU present. \\
			\hline
			Expected Result & The exception should be handled and a useful error message should be displayed. \\
			\hline
			Actual Result & Expected Result \\
			\hline
			Test Status & Pass \\
			\hline
		\end{tabular}
		
		\vspace{5mm}

		Evidence:
		\begin{figure}[h!]
		\centering
		\frame{\includegraphics[width=1\textwidth]{./project-report/src/images/load-model-use-gpu-validation.png}}
		\caption{Use GPU Validation evidence}
		\end{figure}

		Link to video evidence: \url{https://github.com/mcttn22/school-project/blob/main/project-report/testing-videos.md/#use-gpu-validation-1}
\end{itemize}

\pagebreak

\subsubsection{Test Frames}
\label{sec:test-frames-input-validation}
\begin{itemize}
	\item Taken Trained Model Name Validation: \newline\newline
		\begin{tabular}{|p{0.25\linewidth}|p{0.75\linewidth}|}
			\hline
			Description & Try to save a trained model with an already taken name. \\
			\hline
			Data Value & "test" \\
			\hline
			Data Type & Erroneous \\
			\hline
			Expected Result & The exception should be handled and a useful error message should be displayed. \\
			\hline
			Actual Result & Expected Result \\
			\hline
			Test Status & Pass \\
			\hline
		\end{tabular}

		\vspace{5mm}

		Evidence:
		\begin{figure}[h!]
		\centering
		\frame{\includegraphics[width=1\textwidth]{./project-report/src/images/taken-trained-model-name-input-validation.png}}
		\caption{Taken Trained Model Name Validation evidence}
		\end{figure}

		Link to video evidence: \url{https://github.com/mcttn22/school-project/blob/main/project-report/testing-videos.md/#taken-trained-model-name-validation}

	\pagebreak
	
	\item Empty Trained Model Name Validation: \newline\newline
		\begin{tabular}{|p{0.25\linewidth}|p{0.75\linewidth}|}
			\hline
			Description & Try to save a trained model with blank name. \\
			\hline
			Data Value & "" \\
			\hline
			Expected Result & The exception should be handled and a useful error message should be displayed. \\
			\hline
			Actual Result & Expected Result \\
			\hline
			Test Status & Pass \\
			\hline
		\end{tabular}

		\vspace{5mm}

		Evidence:
		\begin{figure}[h!]
		\centering
		\frame{\includegraphics[width=1\textwidth]{./project-report/src/images/empty-trained-model-name-input-validation.png}}
		\caption{Empty Trained Model Name Validation evidence}
		\end{figure}

		Link to video evidence: \url{https://github.com/mcttn22/school-project/blob/main/project-report/testing-videos.md/#empty-trained-model-name-validation}

	\pagebreak

	\item Invalid Delimiter Trained Model Name Validation: \newline\newline
		\begin{tabular}{|p{0.25\linewidth}|p{0.75\linewidth}|}
			\hline
			Description & Try to save a trained model with a name with incorrect delimiters. \\
			\hline
			Data Value & "test  test" \\
			\hline
			Expected Result & The exception should be handled and a useful error message should be displayed. \\
			\hline
			Actual Result & Expected Result \\
			\hline
			Test Status & Pass \\
			\hline
		\end{tabular}

		\vspace{5mm}

		Evidence:
		\begin{figure}[h!]
		\centering
		\frame{\includegraphics[width=1\textwidth]{./project-report/src/images/invalid-delimiter-trained-model-name-input-validation.png}}
		\caption{Invalid Delimiter Trained Model Name Validation evidence}
		\end{figure}

		\begin{sloppypar}
		Link to video evidence: \url{https://github.com/mcttn22/school-project/blob/main/project-report/testing-videos.md/#invalid-delimiter-trained-model-name-validation}
		\end{sloppypar}
\end{itemize}

\subsection{Automated Testing}

\subsubsection{Unit Tests}

Within the test package, I have written the following unit tests:

\begin{itemize}
    \label{sec:database-unit-tests}
    \item Unit tests for the database in a test\_database.py module:
        \begin{itemize}
            \item test\_database\_structure: \newline\newline
			\begin{tabular}{|p{0.25\linewidth}|p{0.75\linewidth}|}
				\hline
				Description & Test that the database tables are set up correctly. \\
				\hline
				Expected Result & Check that the 'Models' table exists in the database and that the table's info matches the following format: \newline
                [(0, 'Model\_ID', 'INTEGER', 0, None, 1), \newline
                (1, 'Dataset', 'TEXT', 1, None, 0), \newline
                (2, 'File\_Location', 'TEXT', 1, None, 0), \newline
                (3, 'Hidden\_Layers\_Shape', 'TEXT', 1, None, 0), \newline
                (4, 'Learning\_Rate', 'FLOAT', 1, None, 0), \newline
                (5, 'Name', 'TEXT', 1, None, 0), \newline
                (6, 'Train\_Dataset\_Size', 'INTEGER', 1, None, 0), \newline
                (7, 'Use\_ReLu', 'INTEGER', 1, None, 0)] \\
				\hline
				Actual Result & Expected Result \\
				\hline
				Test Status & Pass \\
				\hline
			\end{tabular}

			\vspace{5mm}

			\item test\_not\_null\_constraint: \newline\newline
			\begin{tabular}{|p{0.25\linewidth}|p{0.75\linewidth}|}
				\hline
				Description & Test that the NOT NULL constraint is setup. \\
				\hline
				Data Value & ("Test\_Dataset", \newline
                     f"school\_project/saved-models/{uuid.uuid4().hex}.npz", \newline
                     "100, 100", \newline
                     0.1, \newline
                     "Test\_Name", \newline
                     100) \\
				\hline
				Data Type & Erroneous \\
				\hline
				Expected Result & A sqlite3.IntegrityError should be raised. \\
				\hline
				Actual Result & Expected Result \\
				\hline
				Test Status & Pass \\
				\hline
			\end{tabular}

			\vspace{5mm}
			
			\item test\_unique\_constraint: \newline\newline
			\begin{tabular}{|p{0.25\linewidth}|p{0.75\linewidth}|}
				\hline
				Description & Test that the UNIQUE (Dataset, Name) constraint is setup. \\
				\hline
				Data Value & ("Test\_Dataset", \newline
                     f"school\_project/saved-models/{uuid.uuid4().hex}.npz", \newline
                     "100, 100", \newline
                     0.1, \newline
                     "Test\_Name", \newline
                     100, \newline
                     True) \\
				\hline
				Data Type & Erroneous \\
				\hline
				Expected Result & A sqlite3.IntegrityError should be raised. \\
				\hline
				Actual Result & Expected Result \\
				\hline
				Test Status & Pass \\
				\hline
			\end{tabular}

			\vspace{5mm}
			
			\item test\_save\_load\_consistency: \newline\newline
			\begin{tabular}{|p{0.25\linewidth}|p{0.75\linewidth}|}
				\hline
				Description & Test that data is not changed between saving and loading. \\
				\hline
				Data Value & ("Test\_Dataset", \newline
                      f"school\_project/saved-models/{uuid.uuid4().hex}.npz", \newline
                      "100, 100", \newline
                      0.1, \newline
                      "Test\_Name", \newline
                      100, \newline
                      True) \\
				\hline
				Data Type & Normal \\
				\hline
				Expected Result & Data is not changed between saving and loading. \\
				\hline
				Actual Result & Expected Result \\
				\hline
				Test Status & Pass \\
				\hline
			\end{tabular}

			\vspace{5mm}

			\item Evidence:
                \inputminted{python}{./school_project/test/test_database.py}

				\pagebreak

				\begin{figure}[h!]
				\centering
				\frame{\includegraphics[width=1.3\textwidth]{./project-report/src/images/test-database.png}}
				\caption{Unit tests for the database in a test\_database.py module evidence}
				\end{figure}

				Link to video evidence: \url{https://github.com/mcttn22/school-project/blob/main/project-report/testing-videos.md/#test_databasepy}
		\end{itemize}
    
    \label{sec:models-utils-unit-tests}
    \item Unit tests for the utils subpackage of both the cpu and gpu subpackage of the models package. Similarly to the code for the cpu and gpu subpackage, it is 
          not shown as they are identical part from the call to NumPy for the CPU and CuPy for the GPU.
        \begin{itemize}
            \item test\_model.py module:
				\begin{itemize}
					\item test\_train\_dataset\_size: \newline
					\begin{adjustwidth}{-\leftmargin}{0pt}
					\begin{tabular}{|p{0.25\linewidth}|p{0.75\linewidth}|}
						\hline
						Description & Test the size of training dataset to be value chosen. \\
						\hline	
						Data Value & hidden\_layers\_shape = [100, 100], \newline
								train\_dataset\_size = 4, \newline
								learning\_rate = 0.1, \newline
								use\_relu = True \\
						\hline
						Data Type & Normal \\
						\hline
						Expected Result & The number of columns of the training input matrix should be equal to 4. \\
						\hline
						Actual Result & Expected Result \\
						\hline
						Test Status & Pass \\
						\hline
					\end{tabular}
					\end{adjustwidth}
					
					\vspace{5mm}
					
					\item test\_network\_shape: \newline
					\begin{adjustwidth}{-\leftmargin}{0pt}
					\begin{tabular}{|p{0.25\linewidth}|p{0.75\linewidth}|}
						\hline
						Description & Test the neuron count of each layer to match the set shape of the network. \\
						\hline
						Data Value & hidden\_layers\_shape = [100, 100], \newline
                         	train\_dataset\_size = 4, \newline
                         	learning\_rate = 0.1, \newline
                         	use\_relu = True \\
						\hline
						Data Type & Normal \\
						\hline
						Expected Result & The input neuron count of each layer should match [2, 100, 100, 1]. \\
						\hline
						Actual Result & Expected Result \\
						\hline
						Test Status & Pass \\
						\hline
					\end{tabular}
					\end{adjustwidth}
					
					\vspace{5mm}

					\item test\_learning\_rates: \newline
					\begin{adjustwidth}{-\leftmargin}{0pt}
					\begin{tabular}{|p{0.25\linewidth}|p{0.75\linewidth}|}
						\hline
						Description & Test learning rate of each layer to be the same. \\
						\hline
						Data Value & hidden\_layers\_shape = [100, 100], \newline
                         	train\_dataset\_size = 4, \newline
                         	learning\_rate = 0.1, \newline
                         	use\_relu = True \\
						\hline
						Data Type & Normal \\
						\hline
						Expected Result & The learning rate of each layer should be 0.1. \\
						\hline
						Actual Result & Expected Result \\
						\hline
						Test Status & Pass \\
						\hline
					\end{tabular}
					\end{adjustwidth}
					
					\vspace{5mm}

					\item test\_relu\_model\_transfer\_types: \newline
					\begin{adjustwidth}{-\leftmargin}{0pt}
					\begin{tabular}{|p{0.25\linewidth}|p{0.75\linewidth}|}
						\hline
						Description & Test transfer type of each layer to match whats set. \\
						\hline
						Data Values & hidden\_layers\_shape = [100, 100], \newline
                        	train\_dataset\_size = 4, \newline
                            learning\_rate = 0.1, \newline
                            use\_relu = True \\
						\hline
						Data Type & Normal \\
						\hline
						Expected Result & The transfer type of each layer should follow a pattern of ['relu', 'relu', 'sigmoid']. \\
						\hline
						Actual Result & Expected Result \\
						\hline
						Test Status & Pass \\
						\hline
					\end{tabular}
					\end{adjustwidth}
					
					\vspace{5mm}

					\item test\_sigmoid\_model\_transfer\_types: \newline
					\begin{adjustwidth}{-\leftmargin}{0pt}
					\begin{tabular}{|p{0.25\linewidth}|p{0.75\linewidth}|}
						\hline
						Description & Test transfer type of each layer to match whats set. \\
						\hline
						Data Values & hidden\_layers\_shape = [100, 100], \newline
                        	train\_dataset\_size = 4, \newline
                            learning\_rate = 0.1, \newline
                            use\_relu = False \\
						\hline
						Data Type & Normal \\
						\hline
						Expected Result & The transfer type of each layer should follow a pattern of ['sigmoid', 'sigmoid', 'sigmoid'] \\
						\hline
						Actual Result & Expected Result \\
						\hline
						Test Status & Pass \\
						\hline
					\end{tabular}
					\end{adjustwidth}
					
					\vspace{5mm}

					\item test\_weight\_matrice\_shapes: \newline
					\begin{adjustwidth}{-\leftmargin}{0pt}
					\begin{tabular}{|p{0.25\linewidth}|p{0.75\linewidth}|}
						\hline
						Description & Test that each layer's weight matrix has the same number of columns as the layer's input matrix's number of rows, for the matrice multiplication. \\
						\hline
						Data Values & hidden\_layers\_shape = [100, 100], \newline
							train\_dataset\_size = 4, \newline
							learning\_rate = 0.1, \newline
							use\_relu = True \\
						\hline
						Data Type & Normal \\
						\hline
						Expected Result & Each layer's weight matrix has the same number of columns as the layer's input matrix's number of rows. \\
						\hline
						Actual Result & Expected Result \\
						\hline
						Test Status & Pass \\
						\hline
					\end{tabular}
					\end{adjustwidth}

					\vspace{5mm}

					\item test\_bias\_matrice\_shapes: \newline
					\begin{adjustwidth}{-\leftmargin}{0pt}
					\begin{tabular}{|p{0.25\linewidth}|p{0.75\linewidth}|}
						\hline
						Description & Test that each layer's bias matrix has the same number of rows as the result of the layer's weights and input multiplication, for element-wise addition of the biases. \\
						\hline
						Data Values & hidden\_layers\_shape = [100, 100], \newline
							train\_dataset\_size = 4, \newline
							learning\_rate = 0.1, \newline
							use\_relu = True \newline \\
						\hline
						Data Type & Normal \\
						\hline
						Expected Result & Each layer's bias matrix has the same number of rows as the result of the layer's weights and input multiplication. \\
						\hline
						Actual Result & Expected Result \\
						\hline
						Test Status & Pass \\
						\hline
					\end{tabular}
					\end{adjustwidth}

					\vspace{5mm}

					\item test\_layer\_output\_shapes: \newline
					\begin{adjustwidth}{-\leftmargin}{0pt}
					\begin{tabular}{|p{0.25\linewidth}|p{0.75\linewidth}|}
						\hline
						Description & Test the shape of each layer's activation function's output. \\
						\hline
						Data Values & hidden\_layers\_shape = [100, 100], \newline
                         	train\_dataset\_size = 4, \newline
                         	learning\_rate = 0.1, \newline
                         	use\_relu = True \\
						\hline
						Data Type & Normal \\
						\hline
						Expected Result & The shape of each layer's activation function's output should have the same number of rows as the layer's weight matrix and the same number of columns as the layer's input matrix. \\
						\hline
						Actual Result & Expected Result \\
						\hline
						Test Status & Pass \\
						\hline
					\end{tabular}
					\end{adjustwidth}

					\vspace{5mm}

					\item test\_save\_model: \newline
					\begin{adjustwidth}{-\leftmargin}{0pt}
					\begin{tabular}{|p{0.25\linewidth}|p{0.75\linewidth}|}
						\hline
						Description & Test that the weights and biases are saved correctly. \\
						\hline
						Data Values & hidden\_layers\_shape = [100, 100], \newline
                            train\_dataset\_size = 4, \newline
                            learning\_rate = 0.1, \newline
                            use\_relu = True, \newline
							file\_location = f"school\_project/saved-models/{uuid.uuid4().hex}.npz" \\
						\hline
						Data Type & Normal \\
						\hline
						Expected Result & The weights and biases of each layer should not change between saving and loading. \\
						\hline
						Actual Result & Expected Result \\
						\hline
					\end{tabular}
					\end{adjustwidth}

					\vspace{5mm}

					\item Evidence:
						\inputminted{python}{./school_project/test/models/cpu/utils/test_model.py}

						\pagebreak

						\begin{figure}[h!]
						\centering
						\frame{\includegraphics[width=1.3\textwidth]{./project-report/src/images/test-model.png}}
						\caption{Unit tests for model module evidence}
						\end{figure}
	
						Link to video evidence: \url{https://github.com/mcttn22/school-project/blob/main/project-report/testing-videos.md/#test_modelpy}

				\end{itemize}
            \item test\_tools.py module:
				\begin{itemize}
					\item test\_relu: \newline
					\begin{adjustwidth}{-\leftmargin}{0pt}
					\begin{tabular}{|p{0.25\linewidth}|p{0.75\linewidth}|}
						\hline
						Description & Test ReLu output range to be greater than or equal to zero. \\
						\hline
						Data Values & [-100, 0, 100] \\
						\hline
						Data Type & Boundary \\
						\hline
						Expected Result & The output of the ReLu transfer function should be greater than or equal to zero. \\
						\hline
						Actual Result & Expected Result \\
						\hline
						Test Status & Pass \\
						\hline
					\end{tabular}
					\end{adjustwidth}

					\vspace{5mm}

					\item test\_sigmoid: \newline
					\begin{adjustwidth}{-\leftmargin}{0pt}
					\begin{tabular}{|p{0.25\linewidth}|p{0.75\linewidth}|}
						\hline
						Description & Test sigmoid output range to be within 0-1. \\
						\hline
						Data Values & [-100, 0, 100] \\
						\hline
						Data Type & Boundary \\
						\hline
						Expected Result & The output of Sigmoid transfer function should be between zero and one. \\
						\hline
						Actual Result & Expected Result \\
						\hline
						Test Status & Pass \\
						\hline
					\end{tabular}
					\end{adjustwidth}

					\vspace{5mm}

					\item Evidence:
                		\inputminted{python}{./school_project/test/models/cpu/utils/test_tools.py}

						\pagebreak

						\begin{figure}[h!]
						\centering
						\frame{\includegraphics[width=1.3\textwidth]{./project-report/src/images/test-tools.png}}
						\caption{Unit tests for tools module evidence}
						\end{figure}
	
						Link to video evidence: \url{https://github.com/mcttn22/school-project/blob/main/project-report/testing-videos.md/#test_toolspy}
				\end{itemize}
        \end{itemize}
\end{itemize}

\subsubsection{GitHub Automated Testing}

With the following configuration entered in the .github/workflows/tests.yml file, the unit tests are run automatically on GitHub servers after each commit that is pushed to 
GitHub, and the status of the tests (either passing or failing) can be viewed on the repository's page. This automatic testing allows for a faster workflow and allows me to 
identify which changes (commits) cause issues within the code, allowing for easier maintenance of the project.

\inputminted{yaml}{./.github/workflows/tests.yml}

\subsubsection{Docker}

Basic Dockerfile instructions are included for its use in the README.md file, this allows the project to be quickly run and tested in Docker containers. Below shows the 
contents of the basic Dockerfile:

\inputminted{docker}{./Dockerfile}

\end{document}
