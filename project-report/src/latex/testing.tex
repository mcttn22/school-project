\documentclass[./project-report/src/latex/project-report.tex]{subfiles}

\begin{document}

\maketitle

\section{Testing TODO}

\subsection{Investigation}

\subsubsection{test\_model module}

The test\_model module is contained within the frames package, and contains tkinter frames for testing the trained Artificial Neural Network models for each dataset. 
Each frame displays the results of the testing along with a random selection of incorrect and correct predictions.

\inputminted{python}{./school_project/frames/test_model.py}

Which outputs the following for the MNIST dataset:

\pagebreak

\begin{figure}[h!]
\centering
\frame{\includegraphics[width=1\textwidth]{./project-report/src/images/test-mnist-frame.png}}
\end{figure}

And outputs the following for the Cat Recognition dataset:

\pagebreak

\begin{figure}[h!]
\centering
\frame{\includegraphics[width=1\textwidth]{./project-report/src/images/test-cat-recognition-frame.png}}
\end{figure}

And outputs the following for the XOR dataset:

\pagebreak

\begin{figure}[h!]
\centering
\frame{\includegraphics[width=1\textwidth]{./project-report/src/images/test-xor-frame.png}}
\end{figure}

\subsubsection{Effects of Hyper-Parameters}

\includepdf[pages=-, pagecommand={\thispagestyle{plain}}, scale=0.9]{./project-report/src/pdfs/learning-rate-analysis.pdf}
\includepdf[pages=-, pagecommand={\thispagestyle{plain}}, scale=0.9]{./project-report/src/pdfs/epoch-count-analysis.pdf}
\includepdf[pages=-, pagecommand={\thispagestyle{plain}}, scale=0.9]{./project-report/src/pdfs/train-dataset-size-analysis.pdf}
\includepdf[pages=-, pagecommand={\thispagestyle{plain}}, scale=0.9]{./project-report/src/pdfs/layer-count-analysis.pdf}
\includepdf[pages=-, pagecommand={\thispagestyle{plain}}, scale=0.9]{./project-report/src/pdfs/neuron-count-analysis.pdf}
\includepdf[pages=-, pagecommand={\thispagestyle{plain}}, scale=0.9]{./project-report/src/pdfs/relu-analysis.pdf}
\includepdf[pages=-, pagecommand={\thispagestyle{plain}}, scale=0.9]{./project-report/src/pdfs/cpu-vs-gpu-analysis.pdf}

\subsection{Manual Testing}

\subsubsection{Input Validation Testing TODO} % See table on teams, include images + link to video files on github

The following tests check the input validation of each frames' inputs.

\begin{itemize}
    \item Hyper Parameter Frame:
    \begin{itemize}
        \item Use GPU Validation:
            \begin{itemize}
                \item Description: Select Use GPU checkbox without a GPU present.
                \item Expected Result: The exception should be handled and a usefull error message should be dipslayed.
                \item Actual Result: Expected Result
                \item Test Status: Pass
                \item Evidence:
                    \begin{figure}[h!]
                    \centering
                    \frame{\includegraphics[width=1\textwidth]{./project-report/src/images/create-model-use-gpu-validation.png}}
                    \end{figure}
            \end{itemize}

        \pagebreak

        \item Hidden Layers Shape Validation:
            \begin{itemize}
                \item Description: Enter an invalid hidden layers shape.
                \item Data Value: "test"
                \item Data Type: Erroneous
                \item Expected Result: The exception should be handled and a usefull error message should be dipslayed.
                \item Actual Result: Expected Result
                \item Test Status: Pass
                \item Evidence:
                    \begin{figure}[h!]
                    \centering
                    \frame{\includegraphics[width=1\textwidth]{./project-report/src/images/hidden-layers-shape-input-validation.png}}
                    \end{figure}
            \end{itemize}
    \end{itemize}

    \pagebreak

    \item Load Model Frame:
    \begin{itemize}
        \item Use GPU Validation:
            \begin{itemize}
            \item Description: Select Use GPU checkbox without a GPU present.
            \item Expected Result: The exception should be handled and a usefull error message should be dipslayed.
            \item Actual Result: Expected Result
            \item Test Status: Pass
            \item Evidence:
                \begin{figure}[h!]
                \centering
                \frame{\includegraphics[width=1\textwidth]{./project-report/src/images/load-model-use-gpu-validation.png}}
                \end{figure}
            \end{itemize}
    \end{itemize}

    \pagebreak

    \item Test Frames:
    \begin{itemize}
        \item Taken Trained Model Name Validation:
            \begin{itemize}
                \item Description: Try to save a trained model with an already taken name.
                \item Data Value: "test"
                \item Data Type: Erroneous
                \item Expected Result: The exception should be handled and a usefull error message should be dipslayed.
                \item Actual Result: Expected Result
                \item Test Status: Pass
                \item Evidence:
                    \begin{figure}[h!]
                    \centering
                    \frame{\includegraphics[width=1\textwidth]{./project-report/src/images/taken-trained-model-name-input-validation.png}}
                    \end{figure}
            \end{itemize}

        \pagebreak
        
        \item Empty Trained Model Name Validation:
            \begin{itemize}
                \item Description: Try to save a trained model with blank name.
                \item Data Value: ""
                \item Data Type: Erroneous
                \item Expected Result: The exception should be handled and a usefull error message should be dipslayed.
                \item Actual Result: Expected Result
                \item Test Status: Pass
                \item Evidence:
                    \begin{figure}[h!]
                    \centering
                    \frame{\includegraphics[width=1\textwidth]{./project-report/src/images/empty-trained-model-name-input-validation.png}}
                    \end{figure}
            \end{itemize}
    \end{itemize}
\end{itemize}

\subsection{Automated Testing}

\subsubsection{Unit Tests}

Within the test package, I have written the following unit tests for the utils subpackage of both the cpu and gpu subpackage of the models package. Similarly to the code for the cpu 
and gpu subpackage, it is only worth showing the code for the cpu version as both are very similar in functionality.

\begin{itemize}
    \item test\_model.py module:
        \inputminted{python}{./school_project/test/models/cpu/utils/test_model.py}

        \pagebreak

    \item test\_tools.py module:
        \inputminted{python}{./school_project/test/models/cpu/utils/test_tools.py}
\end{itemize}

\inputminted{python}{./school_project/test/models/cpu.py}

\subsubsection{GitHub Automated Testing}

With the following configuration programmed in the .github/workflows/tests.yml file, the unit tests are run automatically on GitHub servers after each commit that is pushed to GitHub, 
and the status of the tests (either passing or failing) can be viewed on the repository's page. This automatic testing allows for a faster workflow and allows me to identify which changes 
(commits) cause issues within the code, allowing for easier maintenance of the project.

\inputminted{yaml}{./.github/workflows/tests.yml}

\end{document}