\documentclass[./project-report/src/latex/project-report.tex]{subfiles}

\begin{document}

\maketitle

\section{Design}

\subsection{Introduction}

The following design focuses have been made for the project:

\begin{itemize}
    \item The program will support multiple platforms to run on, including Windows and Linux.
    \item The program will use python3 as its main programming language.
    \item I will use an object-orientated approach to the project
    \item I will give an option to use either a Graphics card or a CPU to train the Artificial Neural Networks
\end{itemize}

I will also be using SysML for designing the following diagrams.

\subsection{System Architecture}

\begin{figure}[h!]
\centering
\includegraphics[width=1\textwidth]{./project-report/src/images/system-architecture-diagram.png}
\end{figure}

\pagebreak

\subsection{Class Diagrams}

\subsubsection{UI Class Diagram}

\begin{figure}[h!]
\centering
\includegraphics[width=1\textwidth]{./project-report/src/images/ui-class-diagram.png}
\end{figure}

\subsubsection{Model Class Diagram}

\begin{figure}[h!]
\centering
\includegraphics[width=1\textwidth]{./project-report/src/images/model-class-diagram.png}
\end{figure}

\pagebreak

\subsection{System Flow chart}

\begin{figure}[h!]
\centering
\includegraphics[width=1\textwidth]{./project-report/src/images/system-flow-chart.png}
\end{figure}

\subsection{Algorithms}

Refer to Analysis for the algorithms behind the Artificial Neural Networks

\subsection{Data Structures}

I will use the following data structures in the program:

\begin{itemize}
    \item Matrices to represent the layers and allow for a varied number of neurons in each layer. To represent the Matrices I will use both numpy arrays and cupy 
          arrays.
    \item Dictionaries for loading the default hyper-parameter values from a JSON file.
\end{itemize}

\subsection{File Structure}

I will use the following file structures to store necessary data for the program:

\begin{itemize}
    \item A JSON file for storing the default hyper-parameters for creating a new model for each dataset.
    \item I will store the image dataset files in a 'datasets' directory. The dataset files will either be a compressed archive file (such as .pkl.gz files) or of the 
          Hierarchical Data Format (such as .h5) for storing large datasets with fast retrieval.
    \item I will save a pretrained model TODO
\end{itemize}

\subsection{Database Design TODO}

\subsection{Queries TODO}

\subsection{Human-Computer Interaction TODO}

- Labeled screenshots of UI

\subsection{Hardware Design}

To allow for faster training of an Artificial Neural Network, I will give the option to use a Graphics Card to train the Artificial Neural Network if available. 
I will also give the option to load pretrained weights to run on less computationaly powerfull hardware using just the CPU as standard.

\end{document}