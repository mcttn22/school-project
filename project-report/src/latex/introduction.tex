\documentclass[./project-report/src/latex/project-report.tex]{subfiles}

\begin{document}

\maketitle

\clearpage
\section{Introduction}

Artificial Intelligence is a branch of Computer Science that attempts to mimic human cognition in order to perform tasks, understand data and predict outcomes 
\cite{hunt2014artificial}. Machine Learning is a subfield of Artificial Intelligence that uses statistical algorithms, which take as an input training datasets, to produce 
mathematical models that allow prediction of the outcome of unseen datasets. Deep Learning is a further subfield of Machine Learning that uses Artificial Neural Networks, a 
process of learning from data inspired by the human brain. Artificial Neural Networks can be trained to operate on, "learn", a vast number of problems, such as Image 
Recognition, and have uses across multiple fields, such as medical imaging in hospitals.

\subsection{Project Aims}

This project is an investigation into how Artificial Neural Networks work, the effects of changing the hyper-parameters (such as the shape of the network and the 
learning rate) used to tune the models, and particularly their applications in Image Recognition. To achieve this, I have derived and researched all the fundamental 
theory behind the project, using sources such as IBM's online research \cite{IBMweb}, and developed Neural Networks from first principles without the use of any third-party 
Machine Learning libraries. I have then implemented the Artificial Neural Networks in the domain of Image Recognition, by creating trained models and have allowed for 
experimentation in varying the hyper-parameters of each model to provide for comparison and experimentation between different model performances. A Graphical User Interface 
has been developed to provide a mechanism for a researcher, or interested student, to train and test models and alter key hyper-parameters to explore the effect on 
performance and results.

\subsection{Overview}

Developing an Artificial Neural Network has required understanding the fundamental theoretical maths and algorithms underpinning this important technology which has been 
exciting and challenging. In implementing the network I have focused on an object-orientated approach utilising design approaches such as encapsulation and suitable data 
structures such as doubly linked lists. I have also researched and set up a development environment utilising tools such as GitHub, Jupyter Notebook, Visual Studio Code, 
LaTex and automated unit testing etc. At the start of the project, I was lucky enough to interview an expert in the field of Neural Networks and AI and his guidance was 
valuable in setting the scope of the project and suggesting properties of the network to test. I am particularly pleased with being able to implement the network from the 
first principle maths. The main learning I have taken from this project is that Artificial Neural Networks require tuning to best address a particular problem - they are not a 
one-size-fits-all solution - and that this is a combination of research, experimentation and experience.

\pagebreak

\end{document}
