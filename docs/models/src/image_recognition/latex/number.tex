\documentclass[10pt,a4paper]{article}
\usepackage[utf8]{inputenc}
\usepackage[T1]{fontenc}
\usepackage{amsmath}
\usepackage{amsfonts}
\usepackage{amssymb}
\usepackage{natbib}
\usepackage{graphicx}

\title{Number Image Recognition model theory}
\author{Max Cotton}
\date{}

\begin{document}

\maketitle

\section{Setup}

\begin{itemize}
    \item Load MNIST image datasets from http://yann.lecun.com/exdb/mnist/ (Where each image is of a number from zero to ten)
    \begin{itemize}
        \item Where each image in the input dataset, is made up from 28x28 pixels and each pixel has an RGB value from 0 to 255
        \item Each image's matrice is then 'flattened' into a 1 dimensional array of values, where each element is also divided by 255 (max RGB value) to a number between 0 and 1, to standardize the dataset
        \item The output dataset is also loaded, where each output for each image is an array, where the index represents the number of the image, by having a 1 in the index that matches the number represented and zeros for all other indexes.
    \end{itemize}
    \item There is a training dataset with 60,000 pictures and a test dataset with 10,000 pictures (fewer pictures are needed for testing)
    \item Afterwards, the weights and the bias are all initialised to zero/s
\end{itemize}

\section{Model}
This number image recognition model uses a Perceptron Artifical Neural Network model, with the RGB values as the input array, and uses the sigmoid transfer function to obtain 10 output neurons with values between 0 and 1 (where the output neruon with the greatest value is predicted), for a multi-class classification of one of ten numbers.

\end{document}